{Phalcon -\\ Overview}
\section{Pengenalan}
Phalcon diperkenalkan sebagai salah satu Framework PHP terbaru, yang dikembangkan oleh sekelompok pengembang yang antusias. Phalcon adalah Framework yang digabungkan secara longgar, yang berarti memungkinkannya bisa membuat komponen objek menempel seperti lem, berdasarkan kebutuhan aplikasi.

Phalcon memberikan beberapa fitur yang unik sebagai keunggulan dibandingkan framework yang lain (baik framework tradisional atau yang sering dipakai) di pemrograman PHP. Diantaranya:

\begin{itemize}
 \item Framework yang full-stack open source

 \item User hanya membutuhkan code yang lebih sedikit untuk mendapatkan keuntungan di beberapa komponen

 \item bisa dipakai untuk membuat framework independen seperti yang dibutuhkan. Contohnya, jika kita hanya membutuhkan komponen cache yang dimiliki Phalcon, kita bisa menggunakannya di aplikasi apapun baik yang dibuat PHP atau menggunakan framework lain.

 \item Di sisi developer, mereka memounyai konsep MVC (Model-View-Controller) dan ORM (Object-Relational Modeling), bekerja dengan mudahnya dalam pemrograman Phalcon.
 \end{itemize}

\subsection{Performance}

Perbedaan Framework Phalcon dengan Framework Yii dan Laravel \cite{prokofyeva2017analysis}.

\begin{center}
 \begin{tabular}{ | m{2em} | m{3cm}| m{3cm} | m{3cm} }
 \hline
 - & Yii & Laravel & Phalcon \\ [0.5ex]
 \hline\hline
 Tipe dalam Proyek & Yii adalah spesialis membuat proyek skala besar seperti forums, portals, CMS, RESTful web services, dll. & Laravel biasa digunakan untuk aplikasi berbasis web, laravel terkenal karena sintaks nya yang sangat indah dan canggih & Phalcon diguakan untuk semua variasi proyek \\
 \hline
 Database yang Mendukung & Yii mendukung semua RDBMS dan non-RDBMS & Laravel mendukung semua RDBMS & Phalcon memberikan dukungan secara equal (sama) baik RDBMS maupun non-RDBMS \\
 \hline
 Bahasa Pemrograman & Framework Yii menggunakan bahasa pemrograman PHP saja & Laravel menggunakan bahasa pemrograman PHP dan mengikuti pattern MVC & Phalcon menggunakan bahasa pemrograman PHP dan C \\
 \hline
 Keterjangkauan & Yii cukup baik di gunakan di skala proyek kecil ke menengah & Laravel punya keterjangkauan yang tinggi dalam skala proyek & Phalcon cocok untuk proyek skala menengah \\
 \hline
 Performa & Sedikit lambat & Performa tinggi namun masih dibawah Phalcon & Performa Tinggi \\ [1ex]
 \hline
\end{tabular}
\end{center}


\subsubsection{This is the subsubsection}
Here is some text after the subsubsection.
Here is some text after the subsubsection.
Here is some text after the subsubsection.
Here is some text after the subsubsection.

\paragraph{This is the paragraph}
Here is some normal text.
Here is some normal text.
Here is some normal text.
Here is some normal text.

\section{Tips On Special Section Heads}
Here are some things you can do for a special
section head.

\section[This Version of Section Head will be sent Contents]
{Break Long Section heads\\ with double backslash}
Here is some normal text.
Here is some normal text.
Here is some normal text.

 \section[This show how to explicitly break lines
\string\hfill\string\break\space in Table of Contents]
{Here is a Section Title}
See this section head for information on how to explicitly break lines in
table of contents.

\section{How to get \lowercase{lower case} in section head: \lowercase{$p$}$H$}
Here is some normal text.
Here is some normal text.
Here is some normal text.

\section{How to use a macro that has both upper and lower case parts:
\copy\sectsavebox}
See the top of this file where the definition and box were set.

%% Sending different version of section to running head,
%% so that the size of math is correct in running head:
\markright{Sample macro \VT{\lowercase{xyz}} sent to running head}

\section{Equation}

For optimal vertical spacing, no blank lines before or after
equations
\begin{equation}
\alpha\beta\Gamma\Delta
\end{equation}
as you see here.