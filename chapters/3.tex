\section{Struktur Aplikasi}
Didalam project Framework Phalcon, ada beberapa folder dan file penting dan menjadikan sebuah struktur aplikasi, berikut strukturnya sesuai dengan gambar ~\ref{fig:struktur}
\begin{figure}[h!]
\centerline{\includegraphics[width=0.8\textwidth]
{figures/struktur.JPG}}
\caption{Struktur yang ada didalam folder project Phalcon Framework}
\label{fig:struktur}
\end{figure}
Berikut penjelasan dari setiap struktur folder dan filenya:
\subsection{App}
folder ini berisikan script yang paling penting di project. Sebagai bagian terkomplit dari aplikasi web.
\subsubsection{Config}
folder \textit{config} berfungsi untuk membantu konfigurasi yang pasti dijalankan untuk menjalankan aplikasi secara lancar. Contohnya konfigurasi basis data, \textit{library}, dan \textit{services}.
\subsubsection{Controllers}
Berfungsi untuk memproses \textit{requests} dan menjalankan \textit{response} dari \textit{requests} tersebut.
\subsubsection{Library}
Berisikan \textit{library} dari luar \textit{library} bawaan Phalcon.
\subsubsection{Migrations}
Berisikan semua \textit{file} yang berhubungan dengan migrasi data. Dimana data tersebut juga bisa dipakai di framework lain.
\subsubsection{Models}
Berisikan logika-logika yang dibutuhkan untuk berinteraksi dengan basis data yang dipakai. Biasanya dipakai untuk representasi data.
\subsubsection{Views}
Di folder ini terdapat semua \textit{view} yang berhubungan dengan aplikasi web. Bisa dilihat oleh \textit{end-user} menggunakan bantuan dari \textit{Controller}